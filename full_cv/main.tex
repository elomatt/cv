%% start of file `template.tex'.
%% Copyright 2006-2015 Xavier Danaux (xdanaux@gmail.com).
%
% This work may be distributed and/or modified under the
% conditions of the LaTeX Project Public License version 1.3c,
% available at http://www.latex-project.org/lppl/.


\documentclass[11pt,a4paper,sans]{moderncv}        % possible options include font size ('10pt', '11pt' and '12pt'), paper size ('a4paper', 'letterpaper', 'a5paper', 'legalpaper', 'executivepaper' and 'landscape') and font family ('sans' and 'roman')

% moderncv themes
\moderncvstyle{casual}                             % style options are 'casual' (default), 'classic', 'banking', 'oldstyle' and 'fancy'
\moderncvcolor{burgundy}                           % color options 'black', 'blue' (default), 'burgundy', 'green', 'grey', 'orange', 'purple' and 'red'
%\renewcommand{\familydefault}{\sfdefault}         % to set the default font; use '\sfdefault' for the default sans serif font, '\rmdefault' for the default roman one, or any tex font name

\definecolor{color2}{RGB}{64,64,64}

\nopagenumbers{}                                   % uncomment to suppress automatic page numbering for CVs longer than one page

% character encoding
\usepackage[utf8]{inputenc}                       % if you are not using xelatex or lualatex, replace by the encoding you are using

% adjust the page margins
\usepackage[scale=0.5, left=1cm, right=2cm, top=1cm, bottom=3cm]{geometry}
\usepackage{enumitem}
%\setlength{\hintscolumnwidth}{3cm}                % if you want to change the width of the column with the dates
%\setlength{\makecvtitlenamewidth}{10cm}           % for the 'classic' style, if you want to force the width allocated to your name and avoid line breaks. be careful though, the length is normally calculated to avoid any overlap with your personal info; use this at your own typographical risks...

% personal data
\name{Matthew}{Aldous, MPhys, Ph.D}
\address{Flat 7, Columbus House, Chapel Road}{Southampton, SO14 5BQ}{UK}% optional, remove / comment the line if not wanted; the "postcode city" and "country" arguments can be omitted or provided empty
\phone[mobile]{+44~(0)~7858 948715}                   % optional, remove / comment the line if not wanted; the optional "type" of the phone can be "mobile" (default), "fixed" or "fax"
\email{matt.aldous@mail.com}                               % optional, remove / comment the line if not wanted
\homepage{mattaldous.weebly.com}                   % optional, remove / comment the line if not wanted
\social[linkedin]{matthew-aldous}                  % optional, remove / comment the line if not wanted
\social[github]{elomatt}                           % optional, remove / comment the line if not wanted

% bibliography adjustements (only useful if you make citations in your resume, or print a list of publications using BibTeX)
%   to show numerical labels in the bibliography (default is to show no labels)
\makeatletter\renewcommand*{\bibliographyitemlabel}{\@biblabel{\arabic{enumiv}}}\makeatother
%   to redefine the bibliography heading string ("Publications")
%\renewcommand{\refname}{Articles}

\makeatletter
\renewcommand\@biblabel[1]{\textbullet}
\makeatother

\setlength{\hintscolumnwidth}{3.5cm}

%----------------------------------------------------------------------------------
%            content
%----------------------------------------------------------------------------------
\begin{document}
%-----       resume       ---------------------------------------------------------
\makecvtitle
\vspace{-1.5cm}
\section{Professional Summary}
As an independent researcher with extensive lab experience, I have a deeply-rooted enthusiasm for the evolution of quantum technology tools through innovative engineering, and the strong experimental skills to make it a reality.

\section{Experience as a Researcher}
\cventry{May 2017–present}{Postdoctoral Fellow}{University of Birmingham}{}{}
{Seconded to the National Physical Laboratory, I led a multi-disciplinary team to deliver a next-generation compact optical lattice clock.
\\\textit{Key Achievements}:
\begin{itemize}[leftmargin=1cm]
	\item Design, construction and operation of portable multi-wavelength laser stabilisation system.
	\item Integration of documentation for the whole project.
\end{itemize}
}  % arguments 3 to 6 can be left empty

\cventry{May 2016-May 2017}{Postdoctoral Fellow}{University of Southampton}{}{}
{As a post-doc in the Integrated Atom Chips group I continued supporting the research of the group in which I conducted my PhD research. My responsibilities included optics lab work, vacuum chamber building, supervision of students, and writing grant proposals.
\\\textit{Key Achievements}:
\begin{itemize}[leftmargin=1cm]
	\item Development of new atomic traps optimised for compact and portable inertial sensors.
	\item Continual collaboration with Quantum Technologies Hub partners on integration of micro-atomic systems.
\end{itemize}
} 

\cventry{Oct 2012–May 2016}{Postgraduate Researcher}{University of Southampton}{pursuit of research PhD}{}
{My PhD focused on cold atom technology transfer and on techniques for sealing the vacuum chambers of integrated atom chips.
\\\textit{Key Achievements}:
\begin{itemize}[leftmargin=1cm]
	\item Building the University of Southampton's Integrated Atom Chip fabrication capability.
	\item Represented the group at several domestic and international conferences; communicating our work to our peers and forging or maintaining professional collaborations.
\end{itemize}
}

\cventry{Jun-Oct 2012}{Research Assistant}{University of Exeter}{}{}
{As part of a summer project, I designed and built an automated Magneto-Optical Kerr Effect (MOKE) microscope for use in analysis of nano-antennas and assessment of their performance in magnetisation measurements.
\\\textit{Key Achievements}:
\begin{itemize}[leftmargin=1cm]
	\item Outfitted an optics lab from scratch and used it to realise novel results.
	\item Gained experience working as an effective part of a research team.
\end{itemize}
} 

\cventry{Jun-Sep 2011}{Researcher Assistant}{University of Exeter}{}{}
{In my first summer project, I developed a new image acquisition program using LabVIEW to take automated hyper-spectral stacks of organic samples using Raman spectroscopy techniques.
\\\textit{Key Achievements}:
\begin{itemize}[leftmargin=1cm]
	\item Advanced my understanding of LabVIEW to CLAD level.
\end{itemize}
}

% Publications from a BibTeX file without multibib
%  for numerical labels: \renewcommand{\bibliographyitemlabel}{\@biblabel{\arabic{enumiv}}}% CONSIDER MERGING WITH PREAMBLE PART
\renewcommand{\refname}{Academic Output}
\nocite{*}
\bibliographystyle{habbrvyr}
\bibliography{publications}

%\section{Experience as an Educator}
%\cventry{Oct 2013–May 2016}{1st Year Labs Demonstrator}{University of Southampton}{}{}{}
%\cventry{April 2011-present}{Private Tutor}{Freelance}{}{}{}

\section{Education}
\cventry{Oct 2008-Jun 2012}{MPhys Physics with European Study}{University of Exeter}{}{}{
\textbf{First Class Honours} My Masters thesis explored technologies for improving magnetic imaging techniques following a year spent living and studying at the Universit\'e de Rennes 1.
\\\textit{Key Achievements}:
\begin{itemize}[leftmargin=1cm]
	\item Thorough grounding in thermodynamics, quantum mechanics, electromagnetism, statistics and experimental methods.
	\item Full immersion in a new environment gave me the opportunity to develop fluency in a foreign language.
\end{itemize}}
\cventry{Sep 2006-Jun 2008}{A Levels}{Bournemouth School}{}{}{Physics, French, Mathematics, English Literature, Music (AS)}
\cventry{Sep 2004-Jun 2006}{GCSEs}{Bournemouth School}{}{}{9 GCSEs including Double Science and Mathematics at A*}

\section{Skills}
\cvitem{\textbf{Physics}}{Atomic physics, vacuum science, semiconductor, quantum optics}
\cvitem{\textbf{Computing}}{LabVIEW, \LaTeX, python, C/C++, COMSOL, Linux, Windows}
\cvitem{\textbf{Electronics}}{Printed circuit board design, integration and testing, analog and digital}
\cvitem{\textbf{Communication}}{Scientific writing, public engagement, community outreach}
\cvitem{\textbf{Languages}}{French (Fluent), Spanish (Conversational), Italian (Basic)}

\section{Awards}
\cvitem{2016}{STEM for Britain: Silver Award in Physics}
\cvitem{2012}{The Exeter Award}
\cvitem{2012}{Xpression FM Best Presenter}
\cvitem{2009}{University of Exeter Dean’s Commendation}

\section{Interests}
\cvitem{}{Folk music, cycling and cooking}

\end{document}

%% end of file `template.tex'.